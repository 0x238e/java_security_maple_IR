@INPROCEEDINGS{WCNCW.2019.8902627, 
author={A. {Abderrahmane} and G. {Adnane} and C. {Yacine} and G. {Khireddine}}, 
booktitle={2019 IEEE Wireless Communications and Networking Conference Workshop (WCNCW)}, 
title={Android Malware Detection Based on System Calls Analysis and CNN Classification}, 
year={2019}, 
volume={}, 
number={}, 
pages={1-6}, 
abstract={Android is the most used mobile operating system in the world with two billion monthly active users in May 2017 [1]. Provided by Google, to be used on a multitude of smartphones, tablets and other connected objects. It allows the installation of a variety of applications for messaging, calling, news or video games.This multitude of applications facilitate the user’s life and make the device a database rich in personal information such as phone number, emails, messages, confidential correspondence, etc., serving also as a good information repository for hackers. This gives rise to a wave of malignity, using suspicious applications, usually for profit (surcharged messages, sale of personal information or even tools of pressure and threat), or to satisfy personal desires (curiosity, vandalism …).Our solution proposes a behavioral dynamic analysis of the applications likely to be a source of malignancy. The application will be sent towards a distant server through a user-friendly and simple to use interface. It will be installed and executed with a simulation of a human use. After execution, system calls generated by the Linux kernel are collected, processed, and provided to the neural network model that will be used to predict whether the analyzed applications are malware or goodware. This model is built and refined using an APK database varied between goodware and malware.We used a neural network for automatic learning, and more precisely the Convolutional Neural Network (CNN). Our method uses matrix representation of collected system calls and input to the CNN model, a less expensive representation in memory space and therefore accelerate the process of learning.}, 
keywords={Android;System calls;Malware classification;Convolutional Neural Network}, 
doi={10.1109/WCNCW.2019.8902627}, 
ISSN={null}, 
month={April},}
@inproceedings{Sun:2016:TPM:2976749.2978343,
 author = {Sun, Mingshen and Wei, Tao and Lui, John C.S.},
 title = {TaintART: A Practical Multi-level Information-Flow Tracking System for Android RunTime},
 booktitle = {Proceedings of the 2016 ACM SIGSAC Conference on Computer and Communications Security},
 series = {CCS '16},
 year = {2016},
 isbn = {978-1-4503-4139-4},
 location = {Vienna, Austria},
 pages = {331--342},
 numpages = {12},
 url = {http://doi.acm.org/10.1145/2976749.2978343},
 doi = {10.1145/2976749.2978343},
 acmid = {2978343},
 publisher = {ACM},
 address = {New York, NY, USA},
 keywords = {android, android runtime, information-flow tracking, taint analysis, taintart},
} 
@INPROCEEDINGS{SPW.2018.00031, 
author={Z. {Xu} and C. {Shi} and C. C. {Cheng} and N. Z. {Gong} and Y. {Guan}}, 
booktitle={2018 IEEE Security and Privacy Workshops (SPW)}, 
title={A Dynamic Taint Analysis Tool for Android App Forensics}, 
year={2018}, 
volume={}, 
number={}, 
pages={160-169}, 
abstract={The plethora of mobile apps introduce critical challenges to digital forensics practitioners, due to the diversity and the large number (millions) of mobile apps available to download from Google play, Apple store, as well as hundreds of other online app stores. Law enforcement investigators often find themselves in a situation that on the seized mobile phone devices, there are many popular and less-popular apps with interface of different languages and functionalities. Investigators would not be able to have sufficient expert-knowledge about every single app, sometimes nor even a very basic understanding about what possible evidentiary data could be discoverable from these mobile devices being investigated. Existing literature in digital forensic field showed that most such investigations still rely on the investigator's manual analysis using mobile forensic toolkits like Cellebrite and Encase. The problem with such manual approaches is that there is no guarantee on the completeness of such evidence discovery. Our goal is to develop an automated mobile app analysis tool to analyze an app and discover what types of and where forensic evidentiary data that app generate and store locally on the mobile device or remotely on external 3rd-party server(s). With the app analysis tool, we will build a database of mobile apps, and for each app, we will create a list of app-generated evidence in terms of data types, locations (and/or sequence of locations) and data format/syntax. The outcome from this research will help digital forensic practitioners to reduce the complexity of their case investigations and provide a better completeness guarantee of evidence discovery, thereby deliver timely and more complete investigative results, and eventually reduce backlogs at crime labs. In this paper, we will present the main technical approaches for us to implement a dynamic Taint analysis tool for Android apps forensics. With the tool, we have analyzed 2,100 real-world Android apps. For each app, our tool produces the list of evidentiary data (e.g., GPS locations, device ID, contacts, browsing history, and some user inputs) that the app could have collected and stored on the devices' local storage in the forms of file or SQLite database. We have evaluated our tool using both benchmark apps and real-world apps. Our results demonstrated that the initial success of our tool in accurately discovering the evidentiary data.}, 
keywords={digital forensics;mobile computing;mobile device;app-generated evidence;digital forensic practitioners;dynamic Taint analysis tool;real-world Android apps;benchmark apps;real-world apps;Android app forensics;online app stores;law enforcement investigators;mobile forensic toolkits;automated mobile app analysis tool;forensic evidentiary data;Tools;Forensics;Androids;Humanoid robots;Mobile handsets;Manuals;Databases;Taint Analysis;Android;Digital Forensic}, 
doi={10.1109/SPW.2018.00031}, 
ISSN={null}, 
month={May},}
@INPROCEEDINGS{NGMAST.2014.23, 
author={S. Y. {Yerima} and S. {Sezer} and I. {Muttik}}, 
booktitle={2014 Eighth International Conference on Next Generation Mobile Apps, Services and Technologies}, 
title={Android Malware Detection Using Parallel Machine Learning Classifiers}, 
year={2014}, 
volume={}, 
number={}, 
pages={37-42}, 	 
abstract={Mobile malware has continued to grow at an alarming rate despite on-going mitigation efforts. This has been much more prevalent on Android due to being an open platform that is rapidly overtaking other competing platforms in the mobile smart devices market. Recently, a new generation of Android malware families has emerged with advanced evasion capabilities which make them much more difficult to detect using conventional methods. This paper proposes and investigates a parallel machine learning based classification approach for early detection of Android malware. Using real malware samples and benign applications, a composite classification model is developed from parallel combination of heterogeneous classifiers. The empirical evaluation of the model under different combination schemes demonstrates its efficacy and potential to improve detection accuracy. More importantly, by utilizing several classifiers with diverse characteristics, their strengths can be harnessed not only for enhanced Android malware detection but also quicker white box analysis by means of the more interpretable constituent classifiers.}, 
keywords={Android (operating system);invasive software;learning (artificial intelligence);mobile computing;parallel processing;Android malware detection;parallel machine learning classifiers;mobile malware;open platform;mobile smart devices market;Malware;Androids;Humanoid robots;Feature extraction;Classification algorithms;Training;Accuracy;Android;malware detection;machine learning;data mining;parallel classifiers;static analysis;mobile security}, 
doi={10.1109/NGMAST.2014.23}, 
ISSN={2161-2897}, 
month={Sep.},}
@INPROCEEDINGS{IMIS.2014.28, 
author={S. W. {Hsiao} and S. H. {Hung} and R. {Chien} and C. W. {Yeh}}, 
booktitle={2014 Eighth International Conference on Innovative Mobile and Internet Services in Ubiquitous Computing}, 
title={PasDroid: Real-Time Security Enhancement for Android}, 
year={2014}, 
volume={}, 
number={}, 
pages={229-235}, 
abstract={As the number of Android devices increased dramatically in recent years, the number of malware targeted for Android also has spiked up. Among the security risks brought by or the consequences caused by Android malware, the leakage of sensitive information becomes the majority. Current Android system only offers users a permission granting mechanism in the application installation time. Although some vendors place further mechanisms to let users grant or revoke the permissions at any time, the users still cannot distinguish if any data have been normally transmitted or intentionally leaked by the applications. Based on Taint Droid, we created a real-time security scheme on Android, called PasDroid, to trace dubious data flow and to alert the users with information for judging if a transmission should be allowed or not.}, 
keywords={Android (operating system);invasive software;PasDroid scheme;security enhancement;Android devices;malware;security risks;sensitive information leakage;permission granting mechanism;Taint Droid;Androids;Humanoid robots;Smart phones;Malware;Message systems;Libraries;Android security;TaintDroid;mobile malware;privacy;information leakage}, 
doi={10.1109/IMIS.2014.28}, 
ISSN={null}, 
month={July},}
@inproceedings{inproceedings,
author = {Nguyen Quang Do, Lisa and Ali, Karim and Livshits, Benjamin and Bodden, Eric and Smith, Justin and Murphy-Hill, Emerson},
year = {2017},
month = {05},
pages = {39-42},
title = {Cheetah: Just-in-Time Taint Analysis for Android Apps},
doi = {10.1109/ICSE-C.2017.20}
}