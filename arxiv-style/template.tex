\documentclass{article}

\usepackage[
        bibencoding=utf8, 
        style=alphabetic
    ]{biblatex}

\bibliography{bibliography}

\usepackage{arxiv}

\usepackage[utf8]{inputenc} % allow utf-8 input
\usepackage[T1]{fontenc}    % use 8-bit T1 fonts
\usepackage{hyperref}       % hyperlinks
\usepackage{url}            % simple URL typesetting
\usepackage{booktabs}       % professional-quality tables
\usepackage{amsfonts}       % blackboard math symbols
\usepackage{nicefrac}       % compact symbols for 1/2, etc.
\usepackage{microtype}      % microtypography
\usepackage{lipsum}		% Can be removed after putting your text content



\title{\emph{ApakoHa}: an automated tool using dynamic taint analysis for android security focusing on sensitive data}

%\date{September 9, 1985}	% Here you can change the date presented in the paper title
%\date{} 					% Or removing it

\author{
  Yiwei Yang, Longwen Zhang, Kaiyuan Xu, Zhe Ye \\
  Schools of Information and Science Technology\\
  ShanghaiTech University\\
  Shanghai, SH 201210 \\
  \texttt{yangyw,zhanglw2,xuky,yezhe@shanghaitech.edu.cn} \\
}


\begin{document}
\maketitle

\begin{abstract}
  Privacy protection on android phones is a widely discussed topic nowadays. As the main leaking source, many tools analyzing information flow statically and dynamically. Integrating dynamic taint analysis in the development process enables early detection of potential privacy leakage, which reduces the cost of fixing them.
  In this paper, we present \emph{ApakoHa}, a dynamic taint analysis tool for Android apps that interleaves bug fixing and code development in the VS-code integrated development environment. 
  \emph{ApakoHa} is based on the novel framework of TaintART that makes full use of android ART runtime to get information flow, and computes the more complex results and optimizes the bytecode through soot\ref{sec:FlowDroid}. incrementally later using static analyzing tools. Unlike traditional batch-style static-analysis tools, \emph{ApakoHa} causes minimal disruption to the developer’s workflow. This video
demo showcases the main features of \emph{ApakoHa}: 

\end{abstract}


% keywords can be removed
\keywords{DTA \and Android privacy \and automated tool \and operating system}


\section{Introduction}
Android security has attracted much research attention from both academy 
            and industry recently. Dynamic Taint Analysis(DTA) is a classic analysis to detect
             information flow problems and it has been widely adopted to detect private
             data leaks in Android applications. In this project, we will automate the
             process of taint analysis and provide more detailed information about the dataflow and ICFG on the source code. Thus providing a way for Maple IR to continue to compile.

First, in terms of the performance of the DTA, we refer to the TaintART techniques to utilize the 
            compiler and the register allocation of android ART Runtime. Then, a taint propagation 
            framework is proposed and the correctness of the taint propagation analysis is proved by 
            their paper. After we obtain the information flow and what kind of method the information is leaking, we try to make highlight the code on VS-Code front end,
            Finally, in the backend, the function name and the taint source will be input to soot, a java static analyzing tool to output ICFG and optimize the code automatically  by adopting the methods of eliminating, replacing and moving.

We're actually adopting the method of combing the advantages of dynamic and static taint analysis techniques. The pros and cons are listed as follows:

\begin{table}
  \caption{Sample table title}
   \centering
   \begin{tabular}{llll}
     \toprule
     Type     & &Representatives     & Features \\
     \midrule
           & In-component propagation analysis &LeakMiner/CHEX  & $\sim$100     \\
           &  &Input terminal  & $\sim$100     \\
           & Inter-component propagation analysis &Klieber  & $\sim$10      \\
           &  &Octeau/DroidSafe & $\sim$10      \\
           & & Component and library function propagation analysis & $\sim$10      \\
     Static & & Output terminal & $\sim$10      \\
           & & Output terminal & $\sim$10      \\
           & & Output terminal & $\sim$10      \\
     Dynamic &   & Cell body       & up to $10^6$  \\
     \bottomrule
   \end{tabular}
   \label{tab:table}
 \end{table}


\section{Related workflow}

\subsection{Taintdroid}
TaintDroid \ref{sec:TaintDroid} is a system-wide taint tracking system based on Android 4.2+. It aims to minimize runtime overhead that is the amount of additional instructions needed for the implementation of tracking mechanism and, also to monitor the system for sensitive information leakage.

\subsection{Cheetah}
Despite the demonstrated usefulness of DTA in mobile privacy security, poor performance attained by prototypes is a big problem. A novel optimization methodology for dynamic taint tracking based on just-in-time compilation is presented. First, the taint propagation logic is separated from the program logic precisely to simplifying the complexity of the taint propagation analysis. 
\subsection{ARTist}


\section{Our Inplementation}
\lipsum[5]
\begin{equation}
\xi _{ij}(t)=P(x_{t}=i,x_{t+1}=j|y,v,w;\theta)= {\frac {\alpha _{i}(t)a^{w_t}_{ij}\beta _{j}(t+1)b^{v_{t+1}}_{j}(y_{t+1})}{\sum _{i=1}^{N} \sum _{j=1}^{N} \alpha _{i}(t)a^{w_t}_{ij}\beta _{j}(t+1)b^{v_{t+1}}_{j}(y_{t+1})}}
\end{equation}



\subsection{Principles of finding Information flow}
\lipsum[5]
\subsubsection{Taint sink}
\lipsum[5]
\subsubsection{Taint Propogation}
\lipsum[5]

\subsection{Principles of finding ICFG flow}
\lipsum[5]
\subsubsection{Taint sink}
\lipsum[5]
\subsubsection{Taint Propogation}
\lipsum[5]




\paragraph{Paragraph}
\lipsum[7]

\section{Application Implementation}
\subsection{soot}
\lipsum[8] 
\label{sec:FlowDroid}
The documentation for \verb+natbib+ may be found at
\begin{center}
  \url{http://mirrors.ctan.org/macros/latex/contrib/natbib/natnotes.pdf}
\end{center}
Of note is the command \verb+\citet+, which produces citations
appropriate for use in inline text.  For example,
\begin{verbatim}
   \citet{hasselmo} investigated\dots
\end{verbatim}
produces
\begin{quote}
  Hasselmo, et al.\ (1995) investigated\dots
\end{quote}

\begin{center}
  \url{https://www.ctan.org/pkg/booktabs}
\end{center}

\subsection{Workflow}
\lipsum[5]
\subsubsection{benchmark}
\lipsum[5]

\subsection{Figures}
\lipsum[10] 
See Figure \ref{fig:fig1}. Here is how you add footnotes. \footnote{Sample of the first footnote.}
\lipsum[11] 

\begin{figure}
  \centering
  \fbox{\rule[-.5cm]{4cm}{4cm} \rule[-.5cm]{4cm}{0cm}}
  \caption{Sample figure caption.}
  \label{fig:fig1}
\end{figure}

\subsection{Tables}
\lipsum[12]
See awesome Table~\ref{tab:table}.



\subsection{Lists}
\begin{itemize}
\item Lorem ipsum dolor sit amet
\item consectetur adipiscing elit. 
\item Aliquam dignissim blandit est, in dictum tortor gravida eget. In ac rutrum magna.
\end{itemize}


% \bibliographystyle{unsrt}  
% \bibliography{references}  %%% Remove comment to use the external .bib file (using bibtex).
%%% and comment out the ``thebibliography'' section.


%%% Comment out this section when you \bibliography{references} is enabled.
% \begin{thebibliography}{1}
\subsection{References}
  \begin{description}
    \item[\cite{Sun2016CCS}] A Practical Multi-level Information-Flow Tracking System for Android RunTime
    \item[\cite{ABDE2019WCNCW}] Android Malware Detection Based on System Calls Analysis and CNN Classification
    \item[\cite{Xu2018SPW}] A Dynamic Taint Analysis Tool for Android App Forensics
    \item[\cite{Yerima2014NGMAST}] Android Malware Detection Using Parallel Machine Learning Classifiers
    \item[\cite{Hsiao2014IMIS}] PasDroid: Real-Time Security Enhancement for Android
    \item[\cite{Ngu2017ICSE}] Cheetah: Just-in-Time Taint Analysis for Android Apps
    \item[\cite{Arzt2014SIGPLAN}]  FlowDroid: Precise Context, Flow, Field, Object-Sensitive and Lifecycle-Aware Taint Analysis for Android Apps
  \end{description}
  \label{sec:TaintDroid} 
  
% \end{thebibliography}


\end{document}
