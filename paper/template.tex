\documentclass{article}

\usepackage[
        bibencoding=utf8, 
        style=alphabetic
    ]{biblatex}

\bibliography{bibliography}

\usepackage{arxiv}

\usepackage[utf8]{inputenc} % allow utf-8 input
\usepackage[T1]{fontenc}    % use 8-bit T1 fonts
\usepackage{hyperref}       % hyperlinks
\usepackage{url}            % simple URL typesetting
\usepackage{booktabs}       % professional-quality tables
\usepackage{amsfonts}       % blackboard math symbols
\usepackage{nicefrac}       % compact symbols for 1/2, etc.
\usepackage{microtype}      % microtypography
\usepackage{lipsum}		% Can be removed after putting your text content
\usepackage{graphicx}


\title{\emph{ApakoHa}: an automated tool using dynamic taint analysis for android security focusing on sensitive data}

%\date{September 9, 1985}	% Here you can change the date presented in the paper title
%\date{} 					% Or removing it

\author{
  Yiwei Yang, Longwen Zhang, Kaiyuan Xu, Zhe Ye \\
  Schools of Information and Science Technology\\
  ShanghaiTech University\\
  Shanghai, SH 201210 \\
  \texttt{yangyw,zhanglw2,xuky,yezhe@shanghaitech.edu.cn} \\
}


\begin{document}
\maketitle

\begin{abstract}
  Privacy protection on android phones is a widely discussed topic nowadays. As the main leaking source, many tools analyzing information flow statically and dynamically. Integrating dynamic taint analysis in the development process enables early detection of potential privacy leakage, which reduces the cost of fixing them.
  In this paper, we present \emph{ApakoHa}, a dynamic taint analysis tool for Android apps that interleaves bug fixing and code development in the VS-code integrated development environment. 
  \emph{ApakoHa} is based on the novel framework of TaintART that makes full use of android ART runtime to get information flow, and computes the more complex results and optimizes the bytecode through soot\ref{sec:FlowDroid}. incrementally later using static analyzing tools. Unlike traditional batch-style static-analysis tools, \emph{ApakoHa} causes minimal disruption to the developer’s workflow. This video
demo showcases the main features of \emph{ApakoHa}: 

\end{abstract}


% keywords can be removed
\keywords{DTA \and Android privacy \and automated tool \and operating system}


\section{Introduction}
Android security has attracted much research attention from both academy 
            and industry recently. Dynamic Taint Analysis(DTA) is a classic analysis to detect
             information flow problems and it has been widely adopted to detect private
             data leaks in Android applications. In this project, we will automate the
             process of taint analysis and provide more detailed information about the dataflow and ICFG on the source code. Thus providing a way for Maple IR to continue to compile.

First, in terms of the performance of the DTA, we refer to the TaintART techniques to utilize the 
            compiler and the register allocation of android ART Runtime. Then, a taint propagation 
            framework is proposed and the correctness of the taint propagation analysis is proved by 
            their paper. After we obtain the information flow and what kind of method the information is leaking, we try to make highlight the code on VS-Code front end,
            Finally, in the backend, the function name and the taint source will be input to soot, a java static analyzing tool to output ICFG and optimize the code automatically  by adopting the methods of eliminating, replacing and moving.

We're actually adopting the method of combing the advantages of dynamic and static taint analysis techniques. Through static analysis can sort out the general idea, through dynamic analysis can get the actual execution process of the program. Both of them can help each other to realize the deep penetration test of the responsible app. The pros and cons of the state of art are listed as follows:

\begin{table}
  \caption{The comparison of the state of the art android taint analysis}
   \centering
   \begin{tabular}{llll}
     \toprule
     Type     & Representative works &     & Features \\
     \midrule
           & In-component propagation analysis &LeakMiner/CHEX  & Incrementally add callback function \\
            &                                   &                 &  Implementing the Android semantic equivalent model     \\
            &                                   &                 &  Add processing callback function Virtual access point    \\
      
           &  & &     \\
      Static & Inter-component propagation analysis &Klieber  &Match with custom inference rules    \\
           &  &Heros/DroidSafe & Transform data flow analysis to improve accuracy     \\
           &  & &     \\
           &  Component and library & FlowDroid& Manual analysis and Implementation \\
           & function propagation analysis & &    Automatic derivation with flowdroid   \\
           &  & &     \\
     Dynamic & Multi level propagation analysis&TaintDroid & Address multiple levels of communication strategy      \\
           &   & Appsplaygroud...   & Optimization or application extension  \\
     \bottomrule
   \end{tabular}
   \label{tab:table}
 \end{table}


\section{Related workflow}

\subsection{Taintdroid}
TaintDroid \ref{sec:TaintDroid} is a system-wide taint tracking system based on Android 4.2+. It aims to minimize runtime overhead that is the amount of additional instructions needed for the implementation of tracking mechanism and, also to monitor the system for sensitive information leakage.


\begin{figure}[ht]
  \centering
  \includegraphics[scale=0.6]{TaintDroid.png}
  \caption{TaintDroid Multi-level tracking.}
  \label{fig:TaintDroid}
\end{figure}

As shown in Figure \ref{fig:TaintDroid}, the system tracks information at multiple levels. The variable-level tracking means the information flow among memory like stack register and heap object. The method-level tracking means the whenever a method returns, the return value if any should properly propagate to the caller method. The file-level tracking means that the system store taint tag on file system permanently. The message-level tracking means the information flow between processes. An Android application typically uses the broadcast receiver and intent to communicate with each other
.

TaintDroid uses technique call adjacent memory storage for the taint storage. This means taint tag locates next to the memory which the tag is associated with. In theory, this implementation should double both the total memory and the address of object. This eases the calculation of the address of the taint tag.


The taint tag used by TaintDroid is bit-wise, meaning each bit of the allocated memory represents the absence or presence of certain type information. The taint tag is stored in a 32-bit integer. As a result, only 32 types of information can be distinguished. There are several modifications made by TiantDroid developers for the accommodation of taint tag.

1. Stack. Next to each virtual register which is just a 4-byte memory, an additional 4-byte memory is allocated so that during the execution of a method, the taint tag can be stored locally for the method.

2. Calling convention. When a method is invoked or returned, a modified calling convention is used to accommodate the taint tag for the return result.

3. Parcel. This change allows taint tag travel through binder which essentially the inter-process communication procedure.

4. File system. TaintDroid extends the Linux file system so that the metadata on INode can store the taint tags. If such way, a file which receives sensitive information can be marked accordingly. In such way, taint tag can be stored permanently.

With the propagation rules, taint tag can travel through the application and system and when
the information flows to the predefined method or sink, the situation is reported through a special channel and revealed by a front application. For example, a binary operation like addition which takes in virtual register A and B and output to virtual register C. The resulted taint tag of register C should the union of taint tags from register A and B. The union operation in TaintDroid is used as binary "OR" operation. The Table 1 comes from TaintDroid and shows the propagation rules.

\begin{figure}[ht]
  \centering
  \includegraphics[scale=0.4]{TaintDroid2.png}
  \caption{TaintDroid Propagation Rules.}
  \label{fig:TaintDroid}
\end{figure}

TaintDroid is strong for the runtime efficiency. The statistic shows only 14\% runtime overhead occurring during testing. The drawback, however, is that the system is based on Android version 4.2 which is outdated compared today's version 7.0. TaintDroid cannot distinguish files but only label all files under one type of information.


\subsection{Cheetah}
Despite the demonstrated usefulness of DTA in mobile privacy security, poor performance attained by prototypes is a big problem. A novel optimization methodology for taint tracking based on just-in-time static analysis taht discovers and reports the most relevant results to the developer fast is presented. Cheetah\ref{sec:TaintDroid} is a JIT taint analysis for Android applications applying this theory. It enables easy transformation of a distributive dataflow analysis to its correstponding JIT with minimal changes to its fransfer functions. A JIT analysis computes the same dataflow propagations as its base analysis, but it delays some propagations in favor of others by pausing and resuming them later at trigger $stmt$, and each of them is assorciated with a priority that determines the layer at which the JIT analysis resumes its computation.

However, their work is not compatible with android development for from android 7.0 AoT take the place of JIT.
\subsection{Inspeckage}
Inspection is an Xposed module for dynamic analysis of Android apps. Inspection package summarizes many common functions of dynamic analysis and builds a web server. The whole analysis operation can be carried out in a friendly interface environment.

\begin{figure}[ht]
  \centering
  \includegraphics[scale=0.6]{inspeckage1.png}
  \caption{Inspeckage analyzing background.}
  \label{fig:inspeckage1}
\end{figure}

\begin{figure}[ht]
  \centering
  \includegraphics[scale=0.1]{inspeckage2.png}
  \caption{Inspeckage webserver.}
  \label{fig:inspeckage1}
\end{figure}

In theory, it can get all the information by add hooks, but it requires Xposed, which is depreacated since android 7.0.

\section{Theory Principles}
In this part, we'll introduce how the dynamic taint analysis work in tracking the private information and how we find the ICFG graph for those who have possibility to leak information in dynamic runtime which may save a lot of time analyzing non-triggered leakage situation.
\subsection{Principles of finding Information flow in DTA}
\begin{figure}[ht]
  \centering
  \includegraphics[scale=0.3]{TaintART1.png}
  \caption{TaintART ART runtime.}
  \label{fig:TaintART1}
\end{figure}
The main difference between DTA and STA is that DTA has to deal with the runtime. Somtimes hardware support is required. But in android, all the register allocation is based on ART runtime. 



For DTA, the sensitive data to be tracked is called taint source, and the tag added to track sensitive data is called taint tag. When sensitive data is copied or moved to another new location, or data is cleared, its tag will also be copied, moved or cleared, which is called taint logic In taint logic, the status of taint tags that track sensitive data will be stored in taint tag storage. DTA will track the tainted sensitive information and detect whether it is sent out of the system through certain functions, which are called taint sink.

\subsubsection{Runtime support}
The feature of TaintART is that it uses CPU register to store the taint tags. The taint tag is either 1 or 2 bits wide and represents the taint status of other processor registers. During the compilation, the program reserves register 5 and treats it as storage. The register 12 is used for taint propagation.


\subsubsection{Taint Source}
\begin{figure}[ht]
  \centering
  \includegraphics[scale=0.3]{TaintART2.png}
  \caption{TaintART Taint Source.}
  \label{fig:TaintART2}
\end{figure}


\subsubsection{Taint Propogation}
The propagation rules of TaintART shares some similarities with the one from TaintDroid \ref{fig:TaintDroid}. The major difference is TaintART’s rule focus on the instruction at compilation level. HInstruction is a class used by Android for the representation of each Dalvik instruction. The TaintART have rules for every HInstruction so that the resulted binary code can achieve taint propagation as desired. The second major difference is that the unlike binary or operation used by TaintDroid, the TaintART uses a special information merging operation. Take 2-bit wide storage for example. Suppose a taint tag of 0x01 needs to be merged with a taint tag of 0x010, the result would be 0x10. This is because the TaintART uses a level design. When a low-level taint tag encounters a higher-level taint tag, the result would simply be the higher one. The choice is made due to the privacy and runtime efficiency focus.
\begin{figure}[ht]
  \centering
  \includegraphics[scale=0.4]{TaintART3.png}
  \caption{TaintART Taint Propogation.}
  \label{fig:TaintART3}
\end{figure}

Figure \ref{fig:TaintART3} comes from TaintART and shows the operations done for the instruction "MOV R0, R1". The initial taint status represented by R5 is that register 1 and 4 are tainted. The result of MOV operation should leave register 0 tainted. The instruction \#1 masks bit on index 1 which represents the taint tag for register 1 and store the value into temporary register 12 in Instruction \#2. The instruction \#3 shifts the bit to the appropriate position according to the index of result register, which is 0 in the case. Finally, a binary OR operation with input from R5 and R12 gives the result which is then put back to R5. The TaintART modifies the heap to accommodate the taint tag from R5. When a method is invoked, the content of register 5 is saved to stack and then loaded with values according to the taint tags of argument registers.

Under the taint propagation logic, each taint propagation statement is associated with a taint value, which refers to the abstract representation of all pollution state sets that can be observed at some place. The set of all possible taint values is called the taint propagation value range, which is studied abstractly in a holistic way. The taint value range is a product lattice, and the lattice corresponding to each taint variable is shown in Figure \ref{fig:Lattice}.

\begin{figure}[ht]
  \centering
  \includegraphics[scale=0.4]{Lattice.png}
  \caption{Lattice of taint propagation range.}
  \label{fig:Lattice}
\end{figure}
For any value x belongs to the value set V, the intersection operation $\cup$ takes the set combination set $\cup$, which includes: 

(1) x $ \wedge $ x = x, which satisfies the idempotency; 

(2) x $ \wedge $ y = y $ \wedge $ x, which satisfies the exchangeability;

(3) x $ \wedge $ (y $ \wedge $ z) = (x $ \wedge $ y) $ \wedge $z, which satisfies the Associativity. 

The top element is the empty set $\emptyset$, which is expressed as t, which has t$ \wedge $x = x for all x in V; the bottom element is the complete set u, which is expressed as $\perp$, which has $\perp\wedge$ x = $\perp$, for all x in V.

In the taint propagation logic, each statement stat corresponds to a transfer function FS, and the transfer function of a block containing multiple statements can be constructed by combining the transfer functions corresponding to each statement. The function set F is composed of a set of transfer functions $F_{S}$, whose input is the mapping $in[S]$ of the stain variable to the element in the lattice, and the input is a new mapping $out[S]$ after the stain propagation In forward stain propagation analysis, the transfer function $F_{S}$ takes $in[S]$ before the statement as the input and outputs $out[S]$ after the statement; the transfer function f is closed for combination operation, that is, for any function f and G in F, the function h with $H(x) = g(f(x))$ is also in F. there is a unit function I in the transfer function family F, which accepts a mapping as the same mapping returned by the input and output, that is, for All x in V, $I(x) = x$

Monotonicity is defined as: 

(1) for all F and X and Y in all V, $f (x \wedge  y) \leq f (x) \wedge f (y)$. 

Monotonicity can also be equivalently defined as: 

(2) for all F and X and Y in all V, $X \leq y$ implies $f (x) < f (y)$. It is proved that the two definitions are equivalent

It is proved that monotonicity(2) can be derived from monotonicity(1): since $x \wedge y$ is the maximum lower bound of X and y,$ X \wedge y \leq X$ and $X \wedge y \leq y$, from monotonicity(2), $f (x \wedge y) \leq f (x)$ and $f (x \wedge y) \leq f (y)$; at the same time, $f (x \wedge y)$ is the maximum lower bound of $f (x)$ and $f (y)$, monotonicity(1) is proved

Then, it is proved that monotonicity(1) can deduce monotonicity(2): suppose $x \leq y$, from monotonicity(1), $f (x \wedge y) \leq f (x) \wedge f (y)$, according to the definition, $x \wedge y = x$. therefore, $f (x) < f (x) \wedge f (y)$. Because $f (x \wedge y)$ is the maximum lower bound of $f (x)$ and $f (y)$, $f (x) \wedge f (y) < f (y)$, that is, $f (x) < f (y)$, monotonicity(2) is proved

The lattice\ref{fig:Lattice} shown clearly satisfies monotonicity (2) definite For any set X and Y, X belongs to $X \cup Y$.

\subsubsection{Taint sink}
The main choice of the taint slot is the exit function related to the network, and it can provide the log function. The log function is implemented by JNI.
\begin{figure}[ht]
  \centering
  \includegraphics[scale=0.2]{TaintART4.png}
  \caption{Overview of TaintART.}
  \label{fig:TaintART4}
\end{figure}

\subsubsection{Taint tag storage}
Another feature of TaintART is its taint tag representation. Instead of the taint tag used by TaintDroid, the TaintART uses the concept of level. Figure \ref{fig:fig:TaintART2} shows the table from TaintART. There are in total 4 levels where each of them consists some kind of data. Take level 2 as example. It represents both sensor data and location data. The system does not tell what exactly the information a variable carries. If a variable with level 2 information merge with a variable with level 1 information, the result will carry a level 2 tag. This is not desirable for the forensic purpose because of the need to tell the information flow as accurate as possible.

\begin{figure}[ht]
  \centering
  \includegraphics[scale=0.4]{TaintART5.png}
  \caption{Taint tag storage using Register R5.}
  \label{fig:TaintART5}
\end{figure}


\subsection{Principles of finding ICFG}
\lipsum[5]
\subsubsection{Taint sink}

\subsection{Principles of finding IFDS}

\section{Application Implementation}
\subsection{Useful Tools introduction}
\subsubsection{Soot}
Soot is a project with a long history. Its main purpose is to translate Java bytecode and DEX bytecode into an intermediate language. There are four ways to express the output results. Here, we mainly use jimple, which is very similar to Java. It uses less information to express the relationship between classes and variables. It can be described with only 15 opcodes.

Soot is the foundation of the whole static analysis framework. It can disassemble APK files. How to realize it in the middle doesn't need to be concerned for the time being. What the code finally gets is the middle representation. Through the API, you can access the members of each jimple, the call relationship, etc. according to this relationship, you can also generate a visible picture of CFG (control flow graph), which looks comfortable.

The documentation for \verb+Soot+ may be found at  \url{https://github.com/Sable/soot}

\subsection{Workflow}
\lipsum[5]
\subsubsection{benchmark}
\lipsum[5]


% \bibliographystyle{unsrt}  
% \bibliography{references}  %%% Remove comment to use the external .bib file (using bibtex).
%%% and comment out the ``thebibliography'' section.


%%% Comment out this section when you \bibliography{references} is enabled.
% \begin{thebibliography}{1}
\subsection{References}
  \begin{description}
    \item[\cite{Sun2016CCS}] A Practical Multi-level Information-Flow Tracking System for Android RunTime
    \item[\cite{ABDE2019WCNCW}] Android Malware Detection Based on System Calls Analysis and CNN Classification
    \item[\cite{Xu2018SPW}] A Dynamic Taint Analysis Tool for Android App Forensics
    \item[\cite{Yerima2014NGMAST}] Android Malware Detection Using Parallel Machine Learning Classifiers
    \item[\cite{Hsiao2014IMIS}] PasDroid: Real-Time Security Enhancement for Android
    \item[\cite{Ngu2017ICSE}] Cheetah: Just-in-Time Taint Analysis for Android Apps
    \item[\cite{Arzt2014SIGPLAN}]  FlowDroid: Precise Context, Flow, Field, Object-Sensitive and Lifecycle-Aware Taint Analysis for Android Apps
  \end{description}
  \label{sec:TaintDroid} 
  
% \end{thebibliography}


\end{document}
